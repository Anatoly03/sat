\documentclass[12pt, letterpaper]{article}
\usepackage[english]{babel}

\usepackage{amsmath, hhline, listings, multirow, multicol}
\usepackage[table]{xcolor} % http://ctan.org/pkg/xcolor

% fill cell
\newcommand{\fc}{\cellcolor{blue!15}} % \cellcolor{black}

\title{CNF $\cdot$ SAT}
\author{Anatoly Weinstein}

\begin{document}

    \maketitle

    TODO A CNF-SAT is an equation \ldots

    % \subsection*{Definitions}

    % We will call \emph{CNF} as a structure consisting of sets of
    % literal sets. Defining recursively:

    % \begin{center}
    %     \begin{tabular}{c|cc}
    %         Symbol & Type \\ \hline
    %         \textbf{S} & $\{D\}$ & Set of disjunctions \\
    %         \textbf{D} & $\{L\}$ & Set of literals \\
    %         \textbf{L} & $x, \lnot x$ & Positive or negated variable \\
    %         \textbf{x} & $\top, \bot$ & Variable \\
    %     \end{tabular}
    % \end{center}

    \begin{description}

    \item[Conjunctive normalform]
    
    A satisfyability equation in conjunctive
    normalform is a set of literal sets. We
    denote it with $+$ as the logical operation
    \emph{or} and $\cdot$ (can be omitted) as
    \emph{and}.
    \[\{\{a, \overline b\}, \{b, \overline c, \overline d\}, \{\overline a, \overline c\}\}\]
    \[(a+\overline b) \cdot (b +\overline c +\overline d) \cdot (\overline a +\overline c)\]

    \item[Disjunctive normalform]
    
    \[ab + \overline bcd + ac\]

    \end{description}

    \subsection*{Formulas}

    \begin{description}
        \item[Zugzwang.]
        
        If a disjunction consists of one literal, the value of
        this literal is fixed.
        
        \[\{X\} \in CNF \Rightarrow X = \top \]

        \item[Absorption (CNF).]
        
        If a disjunction is a superset of another disjunction, it
        can be ignored.
        
        \[\forall X \subseteq Y : X \cdot Y \equiv_\text{CNF} X \]

        Proof: $(X \Rightarrow Y) \lor \lnot X$ makes $Y$ irrelevant.

        \begin{center}\begin{tabular}{l|c|p{7pt}|}
            \multicolumn{1}{c}{} & \multicolumn{1}{c} X \\ \cline{2-2} \noalign{\vskip\doublerulesep\vskip-\arrayrulewidth} \hhline{~|*{2}{-}}
            \multicolumn{1}{c||} Y & \fc & \\ \hhline{~|*{2}{-}}
            & \fc & \fc \\ \hhline{~|*{2}{-}}
        \end{tabular}\end{center}

        The Absorption Rule allows to freely create new disjunctions,
        as long it's consisting of an already existing disjunction.

        \item[Absorption (DNF).]

        \[\forall X \subseteq Y : X + Y \equiv_\text{DNF} X \]

        Proof: $(X \Rightarrow \top) \land (\lnot X \Rightarrow \lnot Y)$

        \item[Common Contradiction.]
        
        \[(A + X)\cdot(B + \overline X)\equiv_\text{CNF}(BX+A\overline X)\]

        \begin{center}
            \begin{tabular}{l|c|c|c|c|}
                \multicolumn{1}{c}{} & \multicolumn{2}{c} X \\ \cline{2-3} \noalign{\vskip\doublerulesep\vskip-\arrayrulewidth} \hhline{~|*{4}{-}}
                \multicolumn{1}{c||} A && \fc & \fc & \multicolumn{1}{c|}{\fc} \\ \hhline{~|*{4}{-}}
                && \fc && \\ \cline{2-5} \noalign{\vskip\doublerulesep\vskip-\arrayrulewidth} \cline{3-4}
                \multicolumn{2}{c}{} & \multicolumn{2}{c} B \\
            \end{tabular}
        \end{center}

        \item[Resolution.]
        
        \[(A+X)(B+\overline X) \equiv_\text{CNF} (A+X)(B+\overline X)(A+B)\]

        \begin{multicols}{2}
            \begin{center}
                \begin{tabular}{l|c|c|c|c|}
                    \multicolumn{1}{c}{} & \multicolumn{2}{c} X \\ \cline{2-3} \noalign{\vskip\doublerulesep\vskip-\arrayrulewidth} \hhline{~|*{4}{-}}
                    \multicolumn{1}{c||} A && \fc & \fc & \multicolumn{1}{c|}{\fc} \\ \hhline{~|*{4}{-}}
                    && \fc && \\ \cline{2-5} \noalign{\vskip\doublerulesep\vskip-\arrayrulewidth} \cline{3-4}
                    \multicolumn{2}{c}{} & \multicolumn{2}{c} B \\
                \end{tabular}
            \end{center}

            \begin{center}
                \begin{tabular}{l|c|c|c|c|}
                    \multicolumn{1}{c}{} & \multicolumn{2}{c} X \\ \cline{2-3} \noalign{\vskip\doublerulesep\vskip-\arrayrulewidth} \hhline{~|*{4}{-}}
                    \multicolumn{1}{c||} A & {\cellcolor{black!15}} & \fc & \fc & \multicolumn{1}{c|}{\fc} \\ \hhline{~|*{4}{-}}
                    && \fc & {\cellcolor{black!15}} & \\ \cline{2-5} \noalign{\vskip\doublerulesep\vskip-\arrayrulewidth} \cline{3-4}
                    \multicolumn{2}{c}{} & \multicolumn{2}{c} B \\
                \end{tabular}
            \end{center}
        \end{multicols}

        Proof:

        \begin{center}
            $(A+X)(B+\overline X)$ is stronger than $(A+B)$
        \end{center}

        \item[Common Part.]
        
        \[(A + X)\cdot(B + X)\equiv_\text{CNF}(X+AB)\]

        \begin{center}
            \begin{tabular}{l|c|c|c|c|}
                \multicolumn{1}{c}{} & \multicolumn{2}{c} X \\ \cline{2-3} \noalign{\vskip\doublerulesep\vskip-\arrayrulewidth} \hhline{~|*{4}{-}}
                \multicolumn{1}{c||} A &\fc & \fc & \fc & \\ \hhline{~|*{4}{-}}
                & \fc & \fc && \\ \cline{2-5} \noalign{\vskip\doublerulesep\vskip-\arrayrulewidth} \cline{3-4}
                \multicolumn{2}{c}{} & \multicolumn{2}{c} B \\
            \end{tabular}
        \end{center}

        \item[Semi-Common Part.]
        
        \[(AX + B)(A + C) \equiv_\text{CNF} A(X + B) + BC\]

        Proof: \begin{equation}\nonumber\begin{split}
            (AX + B)(A + C) \\
            \equiv_\text{CNF} AX + AXC + AB + BC \\
            \equiv_\text{CNF}^\text{Absorption} AX + AB + BC
        \end{split}\end{equation}

        \item[Common and Contradicting Part]

        \[ (A + B + C) \cdot (A + \lnot B + D) \equiv_\text{CNF} (A + BD + C \overline B)\]

        \begin{center}
            \begin{tabular}{l|p{3px}|p{3px}|p{3px}|p{3px}|r}
                \multicolumn{1}{c}{} & \multicolumn{2}{c} A \\ \cline{2-3} \noalign{\vskip\doublerulesep\vskip-\arrayrulewidth} \hhline{~|*{4}{-}}
                \multicolumn{1}{l||}{\multirow{2}{*}{C}} &\fc & \fc && \fc \\ \hhline{~|*{4}{-}}
                \multicolumn{1}{c||}{} &\fc & \fc & \fc & \fc & \multicolumn{1}{||r}{\multirow{2}{*}{D}} \\ \hhline{~|*{4}{-}}
                &\fc & \fc & \fc && \multicolumn{1}{||r}{} \\ \hhline{~|*{4}{-}}
                & \fc & \fc && \\ \cline{2-5} \noalign{\vskip\doublerulesep\vskip-\arrayrulewidth} \cline{3-4}
                \multicolumn{2}{c}{} & \multicolumn{2}{c} B \\
            \end{tabular}
        \end{center}

        Proof: \begin{equation}\nonumber\begin{split}
            (A + B + C) \cdot (A + \lnot B + D) \\
            \equiv_\text{CNF} (A + (B + C)(\lnot B + D)) \\
            \equiv_\text{CNF} (A + BD + C \overline B)
        \end{split}\end{equation}

    \end{description}

    \section*{Algorithms}

    \subsection*{\texttt{bf0} $\cdot$ Brute Force}

    \begin{enumerate}
        \item Try out every combination of variable interpretations.
    \end{enumerate}

    \subsection*{\texttt{da0} $\cdot$ Disjunctive Absorption}

    \begin{enumerate}
        \item Use trivial optimisations.

        \item Count the number of occurences of every literals
        (positive and negated) and find the literal with most
        occurences.

        \item Branch: Assume the variable and accordingly clean
        up the equation using trivial transformations.

        \item Absorption: Absorb as many disjunctions to the two-
        literals clauses.

        \item If the equation is in 2-CNF-SAT, proceed with the
        next step, otherwise jump to the first step in a recursive
        step.

        \item Solve 2-SAT the equation in polynomial time.
    \end{enumerate}

\end{document}