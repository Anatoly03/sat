\documentclass[12pt, letterpaper]{article}
\usepackage[english]{babel}
\usepackage{xcolor}
\usepackage{amsmath}
\usepackage{listings}

\title{Storing CNF Equations}
\author{Anatoly Weinstein}

\begin{document}

    \maketitle

    This document contains information, such as magic bytes
	and grammars, about how SAT equations will be stored in
	this program.

    \section{Human Readable SAT \texttt{.hsat}}

    \begin{description}
		\item[Magic Bytes]

        The file starts with the bytes \texttt{SAT}

        \item[Grammar]

		Contains a general unformated equation.

		\texttt {
			\textcolor{blue}{S} $\rightarrow$ SAT\string( \textcolor{blue}{EQ} \string) \\
			\textcolor{blue}{EQ} $\rightarrow$ \textcolor{blue}{BINOP}{\string(} \textcolor{blue}{EQ}{,} \textcolor{blue}{EQ} {\string)} \textcolor{blue}{|} {NOT\string(} \textcolor{blue}{EQ} {\string)} \\
			\textcolor{blue}{EQ} $\rightarrow$ \textcolor{blue}{NUM} \textcolor{blue}{|} {T} \textcolor{blue}{|} {F} \\
			\textcolor{blue}{BINOP} $\rightarrow$ {AND} \textcolor{blue}{|} {OR} \textcolor{blue}{|} {XOR} \textcolor{blue}{|} {NOR} \\
		}

    \end{description}

    \section{Compressed CNF-SAT \texttt{.cnf}}

	A compressed CNF-SAT equation consists of three blocks,
	the \texttt{Headers}, \texttt{Solution Strings} and the
	\texttt{Equation String}.

    \begin{description}
		\item[Magic Bytes]
		
		The file starts with the bytes \texttt{CNF}
		followed by a whitespace.

		\item[Comments]
		
		The next block includes comments. These are
		ignored by the parsers. Every comment fills
		one line starting with \texttt{\#} and ending
		with line break.

		\item[Dimension Bytes]
		
		The following two bytes specify the size of
		disjunctions and variable storage.

		The first magic byte specifies the size of
		variables $n$ in bytes. (Constraint: $n \in
		[1,8]$. Note for parser: Reading the symbol
		\texttt{\#} refers to \emph{comments}. See
		above. The first bit is always reserved for
		sign.)

		The following magic byte specifies the size
		of a disjunction $D_\text{max}$ magic bytes.
		(Constraint: $n \in [1,8]$)

		\item[\ldots Solutions]
		
		The first block of the document provides some
		solutions to the logic equations given in the 
		document.

		\begin{description}
			\item[Magic Byte]
			
			Specifies the next state.

			\begin{description}

				\item[\texttt{00} Finished State]
				
				There are no more solutions
				left.

				\item[\texttt{01} Reading Full State]
				
				Next bytes describe a solution.

				\begin{description}
					
					\item[\ldots Solution]
					
					$n$ bits, rounded up to bytes, of
					positional interpretations. A bit
					at position $n$ is $1$, if and
					only of the variable $n$ is truthy.

				\end{description}

				\item[\texttt{02} Reading Semi State]
				
				Next bytes describe a solution.

				\begin{description}
					
					\item[\ldots Solution]
					
					$n$ bits, rounded up to bytes, of
					positional interpretations. A bit
					at position $n$ is $1$, if and
					only of the variable $n$ is truthy.

				\end{description}

				\item[\texttt{FF} Unknown State]
				
				There may be further solutions
				which are not provided in the 
				document.

			\end{description}

		\end{description}
        

		\item[\ldots Equation]
		
		The rest of the document are iterations over
		disjunctions.

		\begin{description}
			\item[Disjunction Size]
			
			Specifies how many variables will
			follow. Size of this magic byte is
			specified in the header.

			Zero to indicate there are no more
			disjunctions.

			\item[\ldots Variables]
			
			Byte String of Variables, every
			variables size is specified in the
			headers. The first bit specifies if
			the variable is negated.

		\end{description}

        \item[Grammar]
        
		{\ }

		\texttt {
			\textcolor{blue}{S} $\rightarrow$ \colorbox{blue!15}{CNF\ } \textcolor{blue}{COMMENTS} \textcolor{blue}{DIMENSION} \textcolor{blue}{SOLUTION} \textcolor{blue}{EQUATION} \\
			\textcolor{blue}{COMMENTS} $\rightarrow$ \colorbox{blue!15}{\#} $\dots$ \textbackslash n \\
			\textcolor{blue}{DIMENSION} $\rightarrow$ $n$$D_{max}$ \\
			\textcolor{blue}{SOLUTION} $\rightarrow$ TODO \\
			\textcolor{blue}{EQUATION} $\rightarrow$ TODO \\
			% \textcolor{blue}{EQ} $\rightarrow$ \textcolor{blue}{BINOP}\string( \textcolor{blue}{EQ}, \textcolor{blue}{EQ} \string) \textcolor{blue}{|} NOT\string( \textcolor{blue}{EQ} \string) \\
			% \textcolor{blue}{EQ} $\rightarrow$ \textcolor{blue}{NUM} \textcolor{blue}{|} T \textcolor{blue}{|} F \\
			% \textcolor{blue}{BINOP} $\rightarrow$ AND \textcolor{blue}{|} OR \textcolor{blue}{|} XOR \textcolor{blue}{|} NOR \\
		}

		\item[Example] Here is an example of a word $w \in $
		CNF divided into sections.
        
		\texttt{
			{\color[HTML]{8f3faf}CNF} \\ \hrule
			{\color[HTML]{00af3f}\# Equation:} {\color[HTML]{af005f}1 2} | {\color[HTML]{5faf00}1 -2 -3} | {\color[HTML]{afafa0}-1 3} \\
			{\color[HTML]{00af3f}\# Solutions:} {\color[HTML]{af5f00}-1 2 -3} | {\color[HTML]{5f00af}1} {\color[HTML]{005faf}3} \\ \hrule
			{\color[HTML]{8f3faf}1 1} \\ \hrule
			{\color[HTML]{af5f00}1 010}00000 \\
			2 {\color[HTML]{5f00af}010}00000 {\color[HTML]{005faf}011}00000 0 \\
			0 \\ \hrule
			2 {\color[HTML]{af005f}00000001 00000010} \\
			3 {\color[HTML]{5faf00}00000001 10000010 10000011} \\
			2 {\color[HTML]{afafa0}10000001 00000011} \\
			0
		}

		The first section consists only of the
		magic bytes \texttt{CNF}. The second section
		consists of comments, every line begins with
		\#. The magic bytes follow.

		The forth divided section consists of the
		solutions. The first solution sets every
		variable and therefore starts with \texttt
		{1}. The interpretation is \texttt{010}
		filled with zeroes. The next solution does
		not specify all variables, therefore starts
		with \texttt{2}. The next bytes till \texttt
		{0} are variables with sign bit.

		The last section is the equation. Every
		conjunction starts with the amount of literals,
		followed by the literal declarations themselves.

    \end{description}
\end{document}