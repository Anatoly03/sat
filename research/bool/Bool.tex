\documentclass[12pt, letterpaper]{article}
\usepackage[english]{babel}
\usepackage{MnSymbol}
\usepackage{amsfonts}
\usepackage[table]{xcolor} % http://ctan.org/pkg/xcolor
\usepackage{hhline}
\usepackage{multirow}
\usepackage{multicol}% http://ctan.org/pkg/multicol
\usepackage{calc}% http://ctan.org/pkg/calc
\usepackage{lipsum}% http://ctan.org/pkg/lipsum

\title{Introduction to Boolean Arithmetics}
\author{Anatoly Weinstein}

\begin{document}

    \maketitle

    In this document we will contain dry theory of boolean equations
    and arithmetics. For the rest of the paper: $0 \in \mathbb{N}$
    holds true.

    \section{Axioms}

    \begin{description}
        \item[Axiom of Binary Choice.] All unknowns $x$ in a
        \emph{boolean system} can take at least one of two values.
        
        \[ \forall x: x \in \mathbb{B} =\{\top,\bot\} \]

        \item[Boolean Systems.] A \emph{boolean system} is a structure,
        which can be mapped to one and only one function $f:\mathbb{N}
        \mapsto\mathbb{B}$.
        
        A boolean system is a \emph{tautology} if its' mapping is
        $f(x)=\top$, a \emph{contradiction}, if $f(x)=\bot$
    \end{description}

    

    \section{Boolean Functions}

    We will call $\Sigma$ the \emph{environment}. Abstractly speaking, the 
    environment of any \emph{boolean system} is the set of all inputs.
    An element $x\in\Sigma$ is an \emph{interpretation}. A
    \emph{finite boolean function} $f:\Sigma\to\mathbb{B}$ with $\Sigma =
    \mathbb{B}^n$, a degree $n\in \mathbb{N}$ and the set of truthy
    interpretations $I \subseteq \Sigma$ is defined as following:

    \begin{equation}
        \nonumber
        x \in \Sigma: f(x) =
        \begin{cases}
            \top, & x \in I \\
            \bot, & x \not\in I
        \end{cases}
        %\Leftrightarrow f(x)=(x\in I)
    \end{equation}

    We say the function is \emph{satisfiable} if $|I| > 0$, 
    \emph{unsatisfiable} or a \emph{contradiction} if $|I| = 0$
    and a \emph{tautology} if $I=\mathcal{P}(X) = \{Y:Y\subseteq X\}$.
    An interpretation $x$ is \emph{truthy} if $f(x) = \top$, and
    \emph{falsy} otherwise. Two boolean functions $f_1, f_2$ are equal
    if and only if their truthy interpretations sets $I_1, I_2$ are equal.

    An \emph{infinite boolean function} $f:{[0,1]}_2\rightarrow\mathbb{B}$
    is a \emph{boolean function} of infinite degree. ${[0,1]}_2$ is the 
    set of all irrational numbers between 0 and 1 in base 2. 

    {\ } % free line

    \textbf{Proof of Truthy.} $\top$ (tautalogy) is \emph{truthy}.
    
    \[f:\emptyset\rightarrow\mathbb{B}\mapsto\top \Rightarrow
    I_f = \{\emptyset\} \Rightarrow |I_f| = 1 \Rightarrow \top\text{ is truthy.}\]

    \textbf{Proof of Falsy.} $\bot$ (contradiction) is \emph{falsy}.

    \[f:\emptyset\rightarrow\mathbb{B}\mapsto\bot \Rightarrow
    I_f = \emptyset \Rightarrow |I_f| = 0 \Rightarrow \bot\text{ is falsy.}\]

    In this document we will mainly work with the 1st-degree function
    \emph{not} (noted as $\lnot$ or overline), and the 2nd-degree functions
    \emph{and} (noted as $\cdot$) and \emph{or} (noted as $+$) defined as:

    \begin{multicols}{3}
        \setlength{\parindent}{0pt}
        \begin{minipage}[t]{\linewidth}%
            \begin{center}
                \begin{tabular}{cp{7pt}p{7pt}}
                    % & \multicolumn{2}{c}{$I=\{\bot\}$} \\
                    $\lnot$ & $x_0$ & {} \\ \cline{2-2} \noalign{\vskip\doublerulesep\vskip-\arrayrulewidth} \hhline{~|*{2}{-}}
                     & \multicolumn{1}{|c|} {\cellcolor{blue!15}} & \multicolumn{1}{c|} {} \\ \hhline{~|*{2}{-}}
                \end{tabular}
            \end{center}
            \[\lnot(x)\in\mathbb{B}\backslash\{x\}\]
        \end{minipage}
        \begin{minipage}[t]{\linewidth}%
            \begin{center}
                \begin{tabular}{cp{7pt}p{7pt}}
                    $+$ & $x_0$ & {} \\ \cline{2-2} \noalign{\vskip\doublerulesep\vskip-\arrayrulewidth} \hhline{~|*{2}{-}}
                    \multicolumn{1}{c||}{$x_1$} & \multicolumn{1}{c|} {\cellcolor{blue!15}} & \multicolumn{1}{c|} {\cellcolor{blue!15}} \\ \hhline{~|*{2}{-}}
                    \multicolumn{1}{c|} {} & \multicolumn{1}{c|} {\cellcolor{blue!15}} & \multicolumn{1}{c|} {} \\ \cline{2-3}
                \end{tabular}
            \end{center}
            \begin{equation}
                \nonumber
                \begin{split}
                    +(x_0x_1)=\top\Leftrightarrow\\ x_0 = \top\lor x_1 = \top
                \end{split}
            \end{equation}
        \end{minipage}
        \begin{minipage}[t]{\linewidth}%
            \begin{center}
                \begin{tabular}{cp{7pt}p{7pt}}
                    $\boldsymbol{\cdot}$ & $x_0$ & {} \\ \cline{2-2} \noalign{\vskip\doublerulesep\vskip-\arrayrulewidth} \hhline{~|*{2}{-}}
                    \multicolumn{1}{c||} {$x_1$} & \multicolumn{1}{c|} {\cellcolor{blue!15}} & \multicolumn{1}{c|} {} \\ \hhline{~|*{2}{-}}
                    \multicolumn{1}{c|} {} & \multicolumn{1}{c|} {} & \multicolumn{1}{c|} {} \\ \cline{2-3}
                \end{tabular}
            \end{center}
            \begin{equation}
                \nonumber
                \begin{split}
                    \cdot(x_0x_1)=\top\Leftrightarrow\\ x_0 = x_1 = \top
                \end{split}
            \end{equation}
        \end{minipage}
    \end{multicols}

    \section{Boolean Equations}

    A \emph{boolean equation} is a \emph{boolean system} with
    \emph{tuple} of variables $X$, the degree $n = |X|$ and the set of
    truthy interpretations $I \subseteq \Sigma = \mathbb{B}^n$. 

    \begin{description}
        
        \item[Boolean Equation.] {
            A \emph{boolean equation} with environment $\Sigma$ and
            binary functions $\mathcal{F}\subseteq\{f:\mathbb{B}^2\rightarrow
            \mathbb{B}\}$ is a string defined by the following context-
            free grammar with macro $\mathcal{M}_X$ creating productions
            for every element of $X$.

            % An example of $\mathcal{F}$ is the set $\{\land, \overline\land, \lor,
            % \rightarrow, \oplus, \text{Id}_\text{Left}\}$

            \begin{center}
                \begin{tabular}{ccc}
                    $S$ & $\rightarrow$& $(S)$ \vline\ $S\ \mathcal{M}_\mathcal{F}\ S$ \vline\ $\lnot (S)$ \vline\ $\mathcal{M}_\mathbb{B}$ \vline\ $\mathcal{M}_\Sigma$\\ 
                \end{tabular}
            \end{center}

            The boolean function for this equation is defined by propagating
            the syntax tree, calling the binary functions while iterating
            over the branches. For every boolean equation $S$ we assign
            $\chi_S(X)$ 

            For the rest of the document we say that $\mathcal{F} = \{+,
            \cdot\}$.
        }
        \item[Conjunctive normal form.] {
            A boolean equation is in conjunctive normal form, when the
            following grammar, too, applies.

            \begin{center}
                \begin{tabular}{ccc}
                    $S$ & $\rightarrow$& $(D)$ \vline\ $S \cdot (D)$ \\ 
                    $D$ & $\rightarrow$& $L$ \vline\ $D + L$ \\ 
                    $L$ & $\rightarrow$& $\mathcal{M}_\Sigma$ \vline\ $\lnot\mathcal{M}_\Sigma$
                \end{tabular}
            \end{center}

            A boolean equation in CNF consists of disjunctions $D$, a
            disjunction consists of literals $L$. The notation $(a+\overline
            b) \cdot (b +\overline c + \overline d) \cdot (\overline a +
            \overline c)$ will be used as type ``set of literal sets'' $\{\{a,
            \lnot b\}, \{b, \lnot c, \lnot d\}, \{\lnot a, \lnot b\}\}$ in
            this paper.
        }
        \item[Notation of CNF.] {
            We will now define some macros for working with CNF-SAT
            equations. Be $S$ a boolean equation in CNF-SAT, consisting
            of disjunctions $D$, $d_i \in D, D_i \subseteq D$, over the
            environment $\Sigma$.

            \[D_i =_S D_j \Leftrightarrow f_{D_i \cdot D} = f_{D_j \cdot D} \]

            Two disjunctions are equivalent over the equation $S$ if 
            under union with every disjunction from $S$, the equations
            are equivalent.
        }
    \end{description}



\end{document}