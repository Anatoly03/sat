\documentclass[12pt, letterpaper]{article}
\usepackage[english]{babel}
\usepackage{amsmath}
\usepackage{hhline}
\usepackage{listings}
\usepackage[table]{xcolor} % http://ctan.org/pkg/xcolor

\title{CNF $\cdot$ SAT}
\author{Anatoly Weinstein}

\begin{document}

    \maketitle

    TODO A CNF-SAT is an equation \ldots

    % \subsection*{Definitions}

    % We will call \emph{CNF} as a structure consisting of sets of
    % literal sets. Defining recursively:

    % \begin{center}
    %     \begin{tabular}{c|cc}
    %         Symbol & Type \\ \hline
    %         \textbf{S} & $\{D\}$ & Set of disjunctions \\
    %         \textbf{D} & $\{L\}$ & Set of literals \\
    %         \textbf{L} & $x, \lnot x$ & Positive or negated variable \\
    %         \textbf{x} & $\top, \bot$ & Variable \\
    %     \end{tabular}
    % \end{center}

    \section{Equations}

    \begin{center}
        \begin{tabular}{cc}
            Set Notation $\cdot$ CNF & $\{\{a, \overline b\}, \{b, \overline c, \overline d\}, \{\overline a, \overline c\}\}$ \\[5px]
            Arithmetic $\cdot$ CNF & $(a+\overline b) \cdot (b +\overline c +\overline d) \cdot (\overline a +\overline c)$ \\[5px]
            Negated $\cdot$ DNF & $\lnot(ab + \overline bcd + ac)$ \\
        \end{tabular}
    \end{center}

    \subsection*{Formulas}

    \begin{description}
        \item[Zugzwang.]
        
        If a disjunction consists of one literal, the value of
        this literal is fixed.
        
        \[\{X\} \in S \Rightarrow X = \top \]

        \item[Absorption.]
        
        If a disjunction is a superset of another disjunction, it
        can be ignored.
        
        \[\forall X \subseteq Y : X \cdot Y \equiv_\text{CNF} X \]

        Proof: $(X \Rightarrow Y) \lor \lnot X$ makes $Y$ irrelevant.

        \begin{center}
            \begin{tabular}{l|c|p{7pt}|}
                \multicolumn{1}{c}{} & \multicolumn{1}{c} X \\ \cline{2-2} \noalign{\vskip\doublerulesep\vskip-\arrayrulewidth} \hhline{~|*{2}{-}}
                \multicolumn{1}{c||} Y & \cellcolor{black} & \\ \hhline{~|*{2}{-}}
                & \cellcolor{black} & \cellcolor{black} \\ \hhline{~|*{2}{-}}
            \end{tabular}
        \end{center}

        The Absorption Rule allows to freely create new disjunctions,
        as long it's consisting of an already existing disjunction.

        \item[Common Contradiction.]
        
        \[(A + X)\cdot(B + \overline X)\equiv_\text{CNF}(BX+A\overline X)\]

        \begin{center}
            \begin{tabular}{l|c|c|c|c|}
                \multicolumn{1}{c}{} & \multicolumn{2}{c} X \\ \cline{2-3} \noalign{\vskip\doublerulesep\vskip-\arrayrulewidth} \hhline{~|*{4}{-}}
                \multicolumn{1}{c||} A && \cellcolor{black} & \cellcolor{black} & \multicolumn{1}{c|}{\cellcolor{black}} \\ \hhline{~|*{4}{-}}
                && \cellcolor{black} && \\ \cline{2-5} \noalign{\vskip\doublerulesep\vskip-\arrayrulewidth} \cline{3-4}
                \multicolumn{2}{c}{} & \multicolumn{2}{c} B \\
            \end{tabular}
        \end{center}

        \item[Common Part.]
        
        \[(A + X)\cdot(B + X)\equiv_\text{CNF}(X+AB)\]

        \begin{center}
            \begin{tabular}{l|c|c|c|c|}
                \multicolumn{1}{c}{} & \multicolumn{2}{c} X \\ \cline{2-3} \noalign{\vskip\doublerulesep\vskip-\arrayrulewidth} \hhline{~|*{4}{-}}
                \multicolumn{1}{c||} A &\cellcolor{black} & \cellcolor{black} & \cellcolor{black} & \\ \hhline{~|*{4}{-}}
                & \cellcolor{black} & \cellcolor{black} && \\ \cline{2-5} \noalign{\vskip\doublerulesep\vskip-\arrayrulewidth} \cline{3-4}
                \multicolumn{2}{c}{} & \multicolumn{2}{c} B \\
            \end{tabular}
        \end{center}

    \end{description}

\end{document}