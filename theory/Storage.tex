\documentclass[12pt, letterpaper]{article}
\usepackage[english]{babel}
\usepackage{xcolor}
\usepackage{amsmath}
\usepackage{listings}

\title{Storing Knf Equations}
\author{Anatoly Weinstein}

\begin{document}

    \maketitle

    This document contains information, such as magic bytes
	and grammars, about how SAT equations will be stored in
	this program.

    \section{Human Readable SAT \texttt{.hsat}}

    \begin{description}
		\item[Magic Bytes] {
        	The file starts with the bytes \texttt{HSAT}
        }
        \item[Grammar] {
        	Contains a general unformated equation.

			\texttt {
				\textcolor{blue}{S} $\rightarrow$ SAT\string( \textcolor{blue}{EQ} \string) \\
				\textcolor{blue}{EQ} $\rightarrow$ \textcolor{blue}{BINOP}\string( \textcolor{blue}{EQ}, \textcolor{blue}{EQ} \string) \textcolor{blue}{|} NOT\string( \textcolor{blue}{EQ} \string) \\
				\textcolor{blue}{EQ} $\rightarrow$ \textcolor{blue}{NUM} \textcolor{blue}{|} T \textcolor{blue}{|} F \\
				\textcolor{blue}{BINOP} $\rightarrow$ AND \textcolor{blue}{|} OR \textcolor{blue}{|} XOR \textcolor{blue}{|} NOR \\
			}
        }
    \end{description}

    \section{Human Readable CNF-SAT \texttt{.hcnf}}

    \begin{description}
		\item[Magic Bytes] {
        	The file starts with the bytes \texttt{HCNF}
        }
        \item[Grammar] {
        	Contains a general unformated equation.

			TODO
        }
    \end{description}

    \section{Compressed CNF-SAT \texttt{.cnf}}

    \begin{description}
		\item[Magic Bytes] {
        	The file starts with the bytes \texttt{CNF}
        }
		\item[Dimension Bytes] {
			The following two bytes specify the size of
			disjunctions and variable storage.

        	The first magic byte specifies the size of
			variables in bytes. (1 byte stores up to 127
			different variables, 2 bytes up to 32,767 
			different variables.)

			The following magic byte specifies the size
			of a disjunction magic bytes. (1 byte allows
			up to 255 variables per disjunction. 2 bytes
			up to 65,536 different variables per
			disjunction)
        }

		\item[\ldots Iteration] {
			The rest of the document are iterations over
			disjunctions.

			\begin{description}
				\item[Disjunction Size] {
					Specifies how many variables will
					follow. Size of this magic byte is
					specified in the header.
				} 
				\item[\ldots Variables] {
					Byte String of Variables, every
					variables size is specified in the
					headers. The first bit specifies if
					the variable is negated.
				} 
			\end{description}
        }
    \end{description}

	
\end{document}