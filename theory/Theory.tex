\documentclass[12pt, letterpaper]{article}
\usepackage[english]{babel}
\usepackage{MnSymbol}
\usepackage{amsfonts}

\title{Sat Solver Optimisations}
\author{Anatoly Weinstein}
\begin{document}
    \maketitle

    This document contains research into optimisations of CNF-SAT problems.

    \section{Definitions}

    \begin{description}
        \item[BOOL] {
            \emph{Boolean Function.}
            A boolean function $f: V \subseteq Variables \rightarrow
            \{0, 1\}$ with truthy sets $P \subseteq \mathcal{P}(Variables)$
            yields $1$ if $V \in P$, else yields $0$.
        }
        \item[SAT] {
            \emph{Boolean Satisfyability Equation.}
            We define a boolean formula by the following context-free
            grammar with start variable $S$, a given variable set
            $Variable = \{x_i : i \in [n-1]\}$ of $n$ variables and a
            set of binary operations $Operation = \{\cdot, +, \oplus,
            \dots\}$ and an assigned boolean function $f_{Operation}
            : \mathcal{P} \subseteq \{L, R\} \rightarrow \{0, 1\}$,
            where $L$ and $R$ is the truthiness of left and right 
            respectively.

            \begin{tabular}{ccc}
                $S$ & $\rightarrow$& $(S)$ \\ 
                $S$ & $\rightarrow$& $S\ Operation\ S$ \\ 
                $S$ & $\rightarrow$& $\lnot S$ \\ 
                $S$ & $\rightarrow$& $Variable$ \\ 
            \end{tabular}
        }
        \item[CNF] {
            \emph{Unmixed conjunctive normal form.} We define a boolean
            formula in conjunctive normal form by the following context-
            free grammar with start variable $S$ and a given variable
            set $Variable = \{x_i : i \in [n-1]\}$ of $n$ variables.

            \begin{tabular}{cccc}
                $S$ & $\rightarrow$& $(Disjunction)$ & \small{Short:} $\mathbf{S}: \{D\}$  \\ 
                $S$ & $\rightarrow$& $S \cdot (Disjunction)$ \\ 
                $Disjunction$ & $\rightarrow$& $Literal$ & \small{Short:} $\mathbf{D}: \{X\}$ \\ 
                $Disjunction$ & $\rightarrow$& $Disjunction + Literal$ \\ 
                $Literal$ & $\rightarrow$& $Variable$ & \small{Short:} $\mathbf{L}: x_i$ \\ 
                $Literal$ & $\rightarrow$& $\lnot Variable$ \\ 
            \end{tabular}

            The notation $a\cdot\overline b + b \cdot\overline c \cdot
            \overline d + \overline a \cdot\overline c$ will be used as 
            a set of literal sets $\{\{a, \lnot b\}, \{b, \lnot c,
            \lnot d\}, \{\lnot a, \lnot b\}\}$ in this paper.
        }
    \end{description}

    \section{Transformation Techniques}

    \begin{description}
        \item[U-SAT] {
            \emph{Unmixed conjunctive normal form.} Given a
            \emph{CNF-SAT} formula, we call it $unmixed$, if
            every disjunction consists of either positive or 
            negated literals.

            A polynomial time reduction $SAT \leq_p U-SAT$
            for an equation $S_E \rightarrow U_E$ would be:

            \[\forall D \in S:\]

            TODO De Morgan
        }
        \item[3-CNF-SAT] {
            \emph{Three-literals conjunctive normal form.} Given a
            \emph{CNF-SAT} formula, we call it $n-SAT$, if
            every disjunction consists of exactly $n$-literals.
            
            TODO Proof
        }
        \item[U3-SAT] {
            \emph{Three-literals unmixed conjunctive normal form.}
            Given a \emph{CNF-SAT} formula, we call it $U3-SAT$, if
            it is $unmixed$ and every disjunction consists of
            exactly $3$-literals.
            
            TODO 
        }
    \end{description}

    \section{Optimisation Techniques}

    \begin{description} {
        \item[Contradiction.]

            A formula with contradictional disjunctions
            is not satisfiable.
            
            \[\forall A : X \cdot \lnot X \cdot A(X) = \bot \]
        }
        \item[Common Literals.] {
            A formula with common literals
            is not satisfiable.
            
            \[\forall A : X \cdot \lnot X \cdot A(X) = \bot \]
        }
    \end{description}
\end{document}