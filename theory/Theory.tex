\documentclass[12pt, letterpaper]{article}
\usepackage[english]{babel}
\usepackage{MnSymbol}
\usepackage{amsfonts}
\usepackage[table]{xcolor} % http://ctan.org/pkg/xcolor
\usepackage{hhline}

\title{Sat Solver Optimisations}
\author{Anatoly Weinstein}
\begin{document}
    \maketitle

    This document contains research into optimisations of CNF-SAT problems.

    % \begin{description}
    %     \item[S] A equation in $CNF-SAT$, consisting of disjunctions $D$
    %     \item[D] A equation in $CNF-SAT$, consisting of literals $L$
    %     \item[L] A variable literal taking either $0$ or $1$
    %     \item[E] A variable literal taking either $0$ or $1$
    % \end{description}

    \section{Definitions}

    \begin{description}
        \item[BOOL] {
            \emph{Boolean Function.}
            A boolean function $f: V \subseteq Variables \rightarrow
            \{\bot, \top\}$ with truthy sets $P \subseteq \mathcal{P}
            (Variables)$ yields $\top$ (truthy) if $V \in P$, else
            yields $\bot$ (falsy).
        }
        \item[SAT] {
            \emph{Boolean Satisfyability Equation.}
            We define a boolean formula by the following context-free
            grammar with start variable $S$, a given variable set
            $Variable = \{x_i : i \in [n-1]\}$ of $n$ variables and a
            set of binary operations $Operation = \{\cdot, +, \oplus,
            \dots\}$ and an assigned boolean function $f_{Operation}
            : \mathcal{P} \subseteq \{L, R\} \rightarrow \{\bot, \top\}$,
            where $L$ and $R$ is the truthiness of left and right 
            respectively.

            \begin{tabular}{ccc}
                $S$ & $\rightarrow$& $(S)$ \\ 
                $S$ & $\rightarrow$& $S\ Operation\ S$ \\ 
                $S$ & $\rightarrow$& $\lnot S$ \\ 
                $S$ & $\rightarrow$& $Variable$ \\ 
            \end{tabular}

            The \emph{boolean equation} is satisfied with interpretation
            $P$, if there is an interpretation $X$ with $f_{SAT}(X) = \top$.
        }
        \item[CNF] {
            \emph{Unmixed conjunctive normal form.} We define a boolean
            formula in conjunctive normal form by the following context-
            free grammar with start variable $S$ and a given variable
            set $Variable = \{x_i : i \in [n-1]\}$ of $n$ variables.

            \begin{tabular}{cccc}
                $S$ & $\rightarrow$& $(Disjunction)$ & \small{Short:} $\mathbf{S}: \{D\}$  \\ 
                $S$ & $\rightarrow$& $S \cdot (Disjunction)$ \\ 
                $Disjunction$ & $\rightarrow$& $Literal$ & \small{Short:} $\mathbf{D}: \{X\}$ \\ 
                $Disjunction$ & $\rightarrow$& $Disjunction + Literal$ \\ 
                $Literal$ & $\rightarrow$& $Variable$ & \small{Short:} $\mathbf{L}: x_i$ \\ 
                $Literal$ & $\rightarrow$& $\lnot Variable$ \\ 
            \end{tabular}

            The notation $(a+\overline b) \cdot (b +\overline c +
            \overline d) \cdot (\overline a +\overline c)$ will be used as 
            a set of literal sets $\{\{a, \lnot b\}, \{b, \lnot c,
            \lnot d\}, \{\lnot a, \lnot b\}\}$ in this paper.
        }
    \end{description}

    \section{Transformation Techniques}

    \begin{description}
        \item[U-SAT] {
            \emph{Unmixed conjunctive normal form.} Given a
            \emph{CNF-SAT} formula, we call it $unmixed$, if
            every disjunction consists of either positive or 
            negated literals.

            A polynomial time reduction $SAT \leq_p U-SAT$
            for an equation $S_E \rightarrow U_E$ would be:

            \[\forall D \in S:\]

            TODO De Morgan
        }
        \item[3-CNF-SAT] {
            \emph{Three-literals conjunctive normal form.} Given a
            \emph{CNF-SAT} formula, we call it $n-SAT$, if
            every disjunction consists of exactly $n$-literals.
            
            TODO Proof
        }
        \item[U3-SAT] {
            \emph{Three-literals unmixed conjunctive normal form.}
            Given a \emph{CNF-SAT} formula, we call it $U3-SAT$, if
            it is $unmixed$ and every disjunction consists of
            exactly $3$-literals.
            
            TODO 
        }
    \end{description}

    \section{Optimisation Techniques}

    \begin{description}
        \item[1L] {
            \emph{Common Literals.}
            If a disjunction consists of one literal, the value of
            this literal is fixed.
            
            \[S = X \cdot A(X) \rightarrow \left(f_{SAT}(Y) \Rightarrow X \in Y \right) \]
        }
        \item[CONTR-D]
        {
            \emph{Contradictional Disjunctions.}

            A formula with contradictional disjunctions
            is not satisfiable.
            
            \[\forall A : X \cdot \lnot X \cdot A(X) = \bot \]
        }
        \item[Common Contradicting Literals.] {
            \emph{Disjunction with common contradictional part.}
            
            \[ (A + B) \cdot (\lnot A + C) \cdot X = (AC + B \overline A) \cdot X\]

            \begin{center}
                \begin{tabular}{lcccc}
                    & \multicolumn{2}{c} A \\ \cline{2-3} \noalign{\vskip\doublerulesep\vskip-\arrayrulewidth} \hhline{~|*{4}{-}}
                    \multicolumn{1}{c||} C & \multicolumn{1}{c|} {\cellcolor{blue!15}} & \multicolumn{1}{c|} {\cellcolor{blue!15}} & \multicolumn{1}{c|} {\cellcolor{blue!15}} & \multicolumn{1}{c|} {} \\ \hhline{~|*{4}{-}}
                    \multicolumn{1}{c|} {} & \multicolumn{1}{c|} {} & \multicolumn{1}{c|} {} & \multicolumn{1}{c|} {\cellcolor{blue!15}} & \multicolumn{1}{c|} {} \\ \cline{2-5} \noalign{\vskip\doublerulesep\vskip-\arrayrulewidth} \cline{3-4}
                    && \multicolumn{2}{c} B \\
                \end{tabular}
            \end{center}
        }
        \item[Common Literals.] {
            \emph{Disjunction with common part.}
            
            \[ (A + B) \cdot (A + C) \cdot X = (A + B C) \cdot X\]

            \begin{center}
                \begin{tabular}{lcccc}
                    & \multicolumn{2}{c} A \\ \cline{2-3} \noalign{\vskip\doublerulesep\vskip-\arrayrulewidth} \hhline{~|*{4}{-}}
                    \multicolumn{1}{c||} C & \multicolumn{1}{c|} {\cellcolor{blue!15}} & \multicolumn{1}{c|} {\cellcolor{blue!15}} & \multicolumn{1}{c|} {\cellcolor{blue!15}} & \multicolumn{1}{c|} {} \\ \hhline{~|*{4}{-}}
                    \multicolumn{1}{c|} {} & \multicolumn{1}{c|} {\cellcolor{blue!15}} & \multicolumn{1}{c|} {\cellcolor{blue!15}} & \multicolumn{1}{c|} {} & \multicolumn{1}{c|} {} \\ \cline{2-5} \noalign{\vskip\doublerulesep\vskip-\arrayrulewidth} \cline{3-4}
                    && \multicolumn{2}{c} B \\
                \end{tabular}
            \end{center}
        }
    \end{description}
\end{document}