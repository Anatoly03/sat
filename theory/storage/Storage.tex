\documentclass[12pt, letterpaper]{article}
\usepackage[english]{babel}
\usepackage{xcolor}
\usepackage{amsmath}
\usepackage{listings}

\title{Storing Knf Equations}
\author{Anatoly Weinstein}

\begin{document}

    \maketitle

    This document contains information, such as magic bytes
	and grammars, about how SAT equations will be stored in
	this program.

    \section{Human Readable SAT \texttt{.hsat}}

    \begin{description}
		\item[Magic Bytes] {
        	The file starts with the bytes \texttt{HSAT}
        }
        \item[Grammar] {
        	Contains a general unformated equation.

			\texttt {
				\textcolor{blue}{S} $\rightarrow$ SAT\string( \textcolor{blue}{EQ} \string) \\
				\textcolor{blue}{EQ} $\rightarrow$ \textcolor{blue}{BINOP}\string( \textcolor{blue}{EQ}, \textcolor{blue}{EQ} \string) \textcolor{blue}{|} NOT\string( \textcolor{blue}{EQ} \string) \\
				\textcolor{blue}{EQ} $\rightarrow$ \textcolor{blue}{NUM} \textcolor{blue}{|} T \textcolor{blue}{|} F \\
				\textcolor{blue}{BINOP} $\rightarrow$ AND \textcolor{blue}{|} OR \textcolor{blue}{|} XOR \textcolor{blue}{|} NOR \\
			}
        }
    \end{description}

    \section{Human Readable CNF-SAT \texttt{.hcnf}}

    \begin{description}
		\item[Magic Bytes] {
        	The file starts with the bytes \texttt{HCNF}
        }
        \item[Grammar] {
        	Contains a general unformated equation.

			TODO
        }
    \end{description}

    \section{Compressed CNF-SAT \texttt{.cnf}}

	A compressed CNF-SAT equation consists of three blocks,
	the \texttt{Headers}, \texttt{Solution Strings} and the
	\text{Equation String}.

    \begin{description}
		\item[Magic Bytes] {
        	The file starts with the bytes \texttt{CNF}
			followed by a whitespace.
        }
		\item[Comments] {
        	The next block includes comments. These are
			ignored by the parsers. Every comment fills
			one line starting with \texttt{\#} and ending
			with line break.
        }
		\item[Dimension Bytes] {
			The following two bytes specify the size of
			disjunctions and variable storage.

        	The first magic byte specifies the size of
			variables $n$ in bytes.

			The following magic byte specifies the size
			of a disjunction $D_\text{max}$ magic bytes.
        }

		\item[\ldots Solutions] {
			The first block of the document provides some
			solutions to the logic equations given in the 
			document.

			\begin{description}
				\item[Magic Byte] {
					Specifies the next state.

					\begin{description}
						\item[\texttt{00}] {
							There are no more solutions
							left.
						} 
						\item[\texttt{01}] {
							Next bytes is a solution.
						} 
						\item[\texttt{FF}] {
							There are further solutions
							which are not provided in the 
							document.
						} 
					\end{description}
				} 
				\item[\ldots Solution] {
					$n$ bits, rounded up to bytes, of
					positional interpretations. A bit
					at position $n$ is $1$, if and
					only of the variable $n$ is truthy.
				} 
			\end{description}
        }

		\item[\ldots Equation] {
			The rest of the document are iterations over
			disjunctions.

			\begin{description}
				\item[Disjunction Size] {
					Specifies how many variables will
					follow. Size of this magic byte is
					specified in the header.

					Zero to indicate there are no more
					disjunctions.
				} 
				\item[\ldots Variables] {
					Byte String of Variables, every
					variables size is specified in the
					headers. The first bit specifies if
					the variable is negated.
				} 
			\end{description}
        }
    \end{description}

	
\end{document}